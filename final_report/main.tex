\documentclass{article}
\usepackage{cite}
\usepackage{hyperref}
\usepackage[onehalfspacing]{setspace}
% \usepackage[bottom=1in, top=1in, left=1in, right=1in]{geometry}

\author{Charlie Gunn and Zach Minot}
\title{Crosshatch: A Web Application for Crossword Collaboration}

\begin{document}
\maketitle

\section{Motivation and Introduction}
\label{mot}
When doing a crossword puzzle from a physical newspaper, it is easy for a few people to gather around and work on solving it together.
This is because all of the clues are visible at once, so each person can scan around for clues they know the answer to. However, this technique
for crossword collaboration does not extend well into the digital world. Most major websites that host interactive crosswords only show a small subset of
clues at a time (e.g. The LA Times \cite{latcrossword}), forcing all collaborators to focus on the same small portion of the clues. In the authors' personal experience,
this is an inferior way to collaborate on a crossword: a lot of the fun of joint crosswords is rapidly going back and forth in different areas
of the grid, calling out newly discovered answers.

To solve this problem, this paper introduces Crosshatch: a web application for crossword collaboration. Crosshatch
allows multiple participants to work on the \textit{same} crossword from their own personal devices (in a way similar to document-sharing services,
but for crosswords instead of general documents). To make finding and solving crosswords easy, Crosshatch provides access to multiple free daily crosswords
from around the US (LA Times, Wall Street Journal, Universal, etc.) and features a convenient user interface for solving.

% enjoyment of crosswords
% crosswords are really fun to do together!
% this is easy in a paper newspaper, but really hard online since only one clue displays at once
% (paper newspaper does not scale to large amounts of people)
\newpage
\subsection{Related Work}
% - paper on collaborative editing. mention how our case is simpler
Crosshatch's core technological design relies on a specific piece of real-time systems technology: collaborative editing.
Collaborative real-time editors are applications that enable seemingly instantaneous, simultaneous editing
of the same digital document by different users.
Possibly the most popular of collaborative editing tools is Google Docs \cite{googledocs}, a collaborative
word document tool similar to Microsoft Office Word.
The complex problem of creating a network structure to facilitate this type of application has been solved with centralized (Client/Server)
and P2P models \cite{p2p} \cite{p2p2}.
Furthermore, different protocols, such as push-based \cite{pushbased} or semi-synchronous \cite{semisynchronous}, can be implemented based on the type of information being edited,
network model constructed, and deadlines for edits to cascade through the network.

However, collaboration on a crossword is much simpler than collaboration on a text file; it requires only
single character inputs at designated points with no formatting or large edits. This eliminates the main challenge of collaborative editing, which is contradictory edits which are non-trivial to merge (e.g., one user deletes the word ``beet'', while another inserts a `g' to make ``beget''). Crosshatch does not have to deal with these sorts of issues due to the nature of single-character grid-based editing -- the server can perform edits in exactly the order it recieves them, with no issues. See Section \ref{editsharing} for more details.

\section{System Architecture}
\subsection{Edit Sharing}
\label{editsharing}

\section{Future Work}
There are many possibly directions for future work, since there are many additionalo features that would be useful in a crossword editing application. A few possibilities are listed below:
\begin{itemize}
	\item Optional incorrect character detection
  \item UI customization (increase crossword size, add colors to letter inputs, customize controls, etc.)
	\item Adding a save-and-quit feature
	\item Adding a chat function
\end{itemize}

\section{Conclusion}

\newpage
\bibliography{sources.bib}
\bibliographystyle{plain}

\end{document}
