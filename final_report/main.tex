\documentclass{article}
\usepackage{cite}
\usepackage{hyperref}
\usepackage[onehalfspacing]{setspace}
% \usepackage[bottom=1in, top=1in, left=1in, right=1in]{geometry}

\author{Charlie Gunn and Zach Minot}
\title{Crosshatch: A Web Application for Crossword Collaboration}

\begin{document}
\maketitle

\section{Motivation and Introduction}
\label{mot}
When doing a crossword puzzle from a physical newspaper, it is easy for a few people to gather around and work on solving it together.
This is because all of the clues are visible at once, so each person can scan around for clues they know the answer to. However, this technique
for crossword collaboration does not extend well into the digital world. Most major websites that host interactive crosswords only show a small subset of
clues at a time (e.g. The LA Times \cite{latcrossword}), forcing all collaborators to focus on the same small portion of the clues. In the authors' personal experience,
this is an inferior way to collaborate on a crossword: a lot of the fun of joint crosswords is rapidly going back and forth in different areas
of the grid, calling out newly discovered answers.

To solve this problem, this paper introduces Crosshatch: a web application for crossword collaboration. Crosshatch
allows multiple participants to work on the \textit{same} crossword from their own personal devices (in a way similar to document-sharing services,
but for crosswords instead of general documents). To make finding and solving crosswords easy, Crosshatch provides access to multiple free daily crosswords
from around the US (LA Times, Wall Street Journal, Universal, etc.) and features a convenient user interface for solving.

% enjoyment of crosswords
% crosswords are really fun to do together!
% this is easy in a paper newspaper, but really hard online since only one clue displays at once
% (paper newspaper does not scale to large amounts of people)
\newpage
\subsection{Related Work}
\label{relatedwork}
% - paper on collaborative editing. mention how our case is simpler
Crosshatch's core technological design relies on a specific piece of real-time systems technology: collaborative editing.
Collaborative real-time editors are applications that enable seemingly instantaneous, simultaneous editing
of the same digital document by different users.
Possibly the most popular of collaborative editing tools is Google Docs \cite{googledocs}, a collaborative
word document tool similar to Microsoft Office Word.
The complex problem of creating a network structure to facilitate this type of application has been solved with centralized (Client/Server)
and P2P models \cite{p2p} \cite{p2p2}.
Furthermore, different protocols, such as push-based \cite{pushbased} or semi-synchronous \cite{semisynchronous}, can be implemented based on the type of information being edited,
network model constructed, and deadlines for edits to cascade through the network.

However, collaboration on a crossword is much simpler than collaboration on a text file; it requires only
single character inputs at designated points with no formatting or large edits. This eliminates the main challenge of collaborative editing,
which is contradictory edits which are non-trivial to merge (e.g., one user deletes the word ``beet'', while another inserts a `g' to make ``beget'').
Crosshatch does not have to deal with these sorts of issues due to the nature of single-character grid-based editing -- the server can perform edits in
exactly the order it receives them, with no issues. See Section \ref{editsharing} for more details.

\section{System Architecture}
\subsection{Basic Structure}

In designing Crosshatch, the first decision to make would be what type of application would it be. The authors' have the most
experience with smartphone apps (particularly, the New York Times Crossword App). However, going this route limits the app to a
specific platform, and the authors wanted to include as much extendability and scalability in this application as possible. This meant instead
of any standalone application, the decision was made to make the app a web application. The authors' also have considerably experience in web applications
than mobile apps, and the coupled with the fact that there are many standardized concepots and libraries for the web reduced the development cost significantly.
Finally, the ability to play on multiple platforms was enticing enough for the authors to make the decision.

Crosshatch was built with four system segments in mind: a frontend user interface, a backend api, a persistent database, 
and a file storage mechanism. These segments were built using multiple libraries, and all containerized within a Docker application.

The frontend library of choice was Vue. This decision was mainly superficial--the authors' wanted to try an updated, state-of-the-art library
that they had never used before. Vue also provided similar concetual programming to React, which both authors have had experience with in the past.

For the backend, the authors chose FastAPI due to its support and familiarity. The authors have had much experience with Python backend APIs
and FastAPI, so it was the default choice for the backend for this application.

PostgresSQL was chosen for the type fo database for similarl reasons to above, but with the added ability for future scalability in the
future if this application were to extend beyond the scope of this class.

The file storage mechanism is temporarily hosted within the backend. This is acceptable, for now, due to the limited amount of crossword the platform supports.
However, in the future, this easy solution will likely need to be replaced with some sort of file storage server on the cloud.

The entire application is compose into a single Docker Application iwth multiple services. The deployed version of Crosshatch is hosted on an AWS EC2 instance.
\subsection{Edit Sharing}
\label{editsharing}

The collaborative crossword architecture was created using the websockets rooms functionality. With websockets, there are rooms,
or a session, that clients can join and listen for events broadcasted to that specific room. In Crosshatch, the granularity is that
for each collaborative crossword thereis a room, and each user is a client that can join any rooms they choose. Each time the crossword is edited,
a single character change event is broadcasted to all other clients, and both the server and the client representation of the crossword is updated
to match the ground truth.

In Google Docs and other colaborative editing tools, text file edit conflicts can become extremely complex as stated in \ref{relatedwork}. However, Crosshatch eliminatesd all
editing conflicts with the limitiation of single character edits. The grid system of the crossword allows for the edit graularity of a single character at a single location on the grid.
Essentially, edit locations are statically determined, and atomic in nature, which allows Crosshatch to consume the edits in order of reception without issue. 

\subsection{Crossword Ingestion}

Another core, but small, part of the application is the ingestion of crosswords from current newspapers. There is no official exposed service
that the authors could find for this functionality, save for a few websites that self hosted the .puz file representations or some major newspapers.
To achieve this, a script was written to scrape the website crosswordifend.com and acquire some of the .puz files from the major newspapers for the current day.
At the moment, this script must be manually run, but hopefully will eventually be automatically scheduled.

\section{Current Functionality and Workflow}

Crosshatch's main functionality is collaborative crossword editing. A user can create a crosshatch session, and then
share the link to others to add other users to the session. To choose a crossword for a session, a user can either choose one of the 
already existing crossword in the database, or upload and title their own crossword in the format of a .puz file. This uploaded .puz file is then added to the database
for all other users to utilize.

% let's add some screenshots here
There are two main screens: the puzzle select screen and the collaboration screen. The puzzle select screen include the user interface for uploading a custom crossword
and selecting a crossword to create a session. To upload a custom crossword, the user need to title the crossword in the input form, upload the .puz file representation of the crossword,
and click submit. After the user refreshes the page, they can see their crossword added to the list of current crosswords in the database. To select a crossword,
the user simply needs to click on the title of the crossword in the list and they will be redirected to their collaborative editing page.

The collaborative editing page showcases only the crossword and the current clue. The current row or column is highlighted on the crossword, and the corresponding clue
is shown below the crossword. At the most basic form, a user can click on a square to edit, and then type a character to place into the square. The user's cursor will
move to the next square in the current clue, or the next clue if there are no more squares in the current clue. Pressing backspace will delete the character from the current square,
and move the cursor or clue backwards similarly.

The editing page also offers a set of keyboard controls. The arrow keys will change the clue from across to down or move the cursor in the coresponding direction. Space will
switch the clue from across to down and vice versa. Delete removes the current letter, but does not move the cursor backwards. Tab will move forward one clue,
while Shift + Tab will move backwards one clue.

\section{Future Work}
There are many possibly directions for future work, since there 
are many additional features that would be useful in a crossword editing application. 
A few possibilities are listed below:
\begin{itemize}
	\item Optional incorrect character detection
  	\item UI customization (increase crossword size, add colors to letter inputs, customize controls, etc.)
	\item Adding a save-and-quit feature
	\item Adding a chat function
\end{itemize}

\section{Conclusion}

\newpage
\bibliography{sources.bib}
\bibliographystyle{plain}

\end{document}
